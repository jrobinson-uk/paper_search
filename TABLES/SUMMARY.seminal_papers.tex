\begin{table*}[t]
\begin{tabular}{lp{2cm}p{2cm}p{2cm}p{2cm}p{2cm}p{2cm}r}
Paper&Koli Calling&International Computing Education Research&Technical Symposium on Computer Science Education&Innovation and Technology in Computer Science Education&Australasian Computing Education&&Total\\\hline
Achievement motivation: Conceptions of ability, subjective experience, task choice, and performance.,Nicholls, John G & 0 & 0 & 0 & 0 & 0 & 0 & 0\\
The theory of affordances,Gibson, James J & 1 & 0 & 0 & 0 & 1 & 0 & 2\\
The organization of behavior. A neuropsychological theory,Hebb, DO & 0 & 0 & 0 & 0 & 0 & 0 & 0\\
Psychology as the behaviorist views it.,Watson, John B & 1 & 0 & 0 & 0 & 0 & 0 & 1\\
Learning and teaching styles in engineering education,Felder, Richard M and Silverman, Linda K and others & 1 & 4 & 0 & 0 & 7 & 0 & 12\\
The magical number seven, plus or minus two: Some limits on our capacity for processing information.,Miller, George A & 3 & 3 & 2 & 0 & 0 & 1 & 9\\
Cognitive apprenticeship: Teaching the craft of reading, writing and mathematics,Collins, Allan and Brown, John Seely and Newman, Susan E & 4 & 2 & 2 & 0 & 3 & 0 & 11\\
Cognitive load during problem solving: Effects on learning,Sweller, John & 2 & 1 & 11 & 0 & 2 & 1 & 17\\
Multimedia learning,Mayer, Richard E & 3 & 3 & 6 & 0 & 1 & 0 & 13\\
Communities of practice: Learning, meaning, and identity,Wenger, Etienne & 5 & 2 & 7 & 2 & 4 & 4 & 24\\
Conditions of learning,Gagn{\'e}, Robert Mills & 1 & 0 & 0 & 0 & 0 & 0 & 1\\
The fundamentals of learning.,Thorndike, Edward L & 0 & 0 & 0 & 0 & 0 & 0 & 0\\
Connectivism: A Learning Theory for the Digital Age,Siemens, George & 0 & 0 & 0 & 0 & 0 & 0 & 0\\
Mindstorms: Children, computers, and powerful ideas,Papert, Seymour & 15 & 8 & 10 & 5 & 12 & 1 & 51\\
Experience and education,Dewey, John & 3 & 2 & 0 & 1 & 4 & 0 & 10\\
Conversation, cognition and learning: A cybernetic theory and methodology,Pask, Gordon & 0 & 0 & 0 & 0 & 0 & 0 & 0\\
The act of discovery,Bruner, Jerome S & 2 & 0 & 0 & 0 & 0 & 0 & 2\\
Mental representations: A dual coding approach,Paivio, Allan & 1 & 0 & 0 & 0 & 2 & 1 & 4\\
Elaboration theory,Reigeluth, C and Stein, R & 0 & 0 & 0 & 0 & 0 & 0 & 0\\
Emotional intelligence,Goleman, Daniel & 0 & 1 & 1 & 0 & 1 & 0 & 3\\
Work and motivation,Vroom, Victor Harold & 1 & 0 & 0 & 0 & 0 & 0 & 1\\
Experiential learning: Experience as the source of learning and development,Kolb, David A & 7 & 7 & 2 & 1 & 3 & 2 & 22\\
The role of deliberate practice in the acquisition of expert performance.,Ericsson, K Anders and Krampe, Ralf T and Tesch-R{\"o}mer, Clemens & 0 & 1 & 3 & 1 & 1 & 1 & 7\\
Flow and the psychology of discovery and invention,Csikszentmihalyi, Mihaly & 2 & 0 & 0 & 1 & 0 & 0 & 3\\
Functional Context Education. Workshop Resource Notebook.,Sticht, Thomas G & 0 & 0 & 0 & 1 & 0 & 0 & 1\\
Gestalt theory.,Wertheimer, Max & 0 & 0 & 0 & 0 & 0 & 0 & 0\\
The magical number seven, plus or minus two: Some limits on our capacity for processing information.,Miller, George A & 3 & 3 & 2 & 0 & 0 & 1 & 9\\
The behavior of organisms,Skinner, Burrhus Frederic & 0 & 0 & 0 & 0 & 0 & 0 & 0\\
Learning from text, levels of comprehension, or: Why anyone would read a story anyway,Kintsch, Walter & 0 & 0 & 0 & 0 & 0 & 0 & 0\\
The present status of interference theory.,Postman, Leo & 0 & 0 & 0 & 0 & 0 & 0 & 0\\
What Is Invitational Education and How Does It Work?.,Purkey, William Watson & 0 & 0 & 0 & 0 & 0 & 0 & 0\\
There is more than one kind of learning.,Tolman, Edward C & 1 & 0 & 0 & 0 & 0 & 0 & 1\\
Learner-centered design: The challenge for HCI in the 21st century,Soloway, Elliot and Guzdial, Mark and Hay, K & 0 & 0 & 2 & 0 & 0 & 0 & 2\\
Learning and teaching styles in engineering education,Felder, Richard M and Silverman, Linda K and others & 1 & 4 & 0 & 0 & 7 & 0 & 12\\
A theory of human motivation.,Maslow, Abraham Harold & 2 & 0 & 0 & 0 & 1 & 0 & 3\\
Mastery learning,Bloom, Benjamin S & 1 & 1 & 4 & 0 & 2 & 1 & 9\\
Mental models in cognitive science,Johnson-Laird, Philip N & 0 & 0 & 0 & 0 & 0 & 0 & 0\\
Metacognition and cognitive monitoring: A new area of cognitive--developmental inquiry.,Flavell, John H & 9 & 7 & 4 & 0 & 16 & 3 & 39\\
A social-cognitive approach to motivation and personality.,Dweck, Carol S and Leggett, Ellen L & 0 & 1 & 3 & 0 & 1 & 0 & 5\\
A split-attention effect in multimedia learning: Evidence for dual processing systems in working memory.,Mayer, Richard E and Moreno, Roxana & 0 & 0 & 1 & 0 & 0 & 0 & 1\\
The behavior of organisms: an experimental analysis.,Skinner, BF & 0 & 0 & 0 & 0 & 0 & 0 & 0\\
Problem-based learning: An approach to medical education,Barrows, Howard S and Tamblyn, Robyn M and others & 0 & 0 & 0 & 0 & 0 & 0 & 0\\
Behavior and personality: Psychological behaviorism,Staats, Walter W & 0 & 0 & 0 & 0 & 0 & 0 & 0\\
A theory of Pavlovian conditioning: Variations in the effectiveness of reinforcement and nonreinforcement,Rescorla, Robert A and Wagner, Allan R and others & 0 & 0 & 0 & 0 & 0 & 0 & 0\\
The role of tutoring in problem solving,Wood, David and Bruner, Jerome S and Ross, Gail & 0 & 0 & 1 & 0 & 5 & 0 & 6\\
Self-determination theory and the facilitation of intrinsic motivation, social development, and well-being.,Ryan, Richard M and Deci, Edward L & 6 & 1 & 1 & 0 & 2 & 0 & 10\\
Approaches to training and development,Laird, Dugan & 0 & 0 & 0 & 0 & 0 & 0 & 0\\
Situated cognition and the culture of learning,Brown, John Seely and Collins, Allan and Duguid, Paul & 5 & 2 & 4 & 1 & 4 & 1 & 17\\
Situated learning: Legitimate peripheral participation,Lave, Jean and Wenger, Etienne and others & 13 & 10 & 12 & 2 & 9 & 1 & 47\\
Mind in society: The development of higher psychological processes,Vygotsky, Lev Semenovich & 15 & 3 & 9 & 1 & 7 & 1 & 36\\
Social learning theory,Bandura, Albert and Walters, Richard H & 0 & 0 & 1 & 0 & 0 & 0 & 1\\
A split-attention effect in multimedia learning: Evidence for dual processing systems in working memory.,Mayer, Richard E and Moreno, Roxana & 0 & 0 & 1 & 0 & 0 & 0 & 1\\
The origins of intelligence in children,Piaget, Jean and Cook, Margaret & 0 & 1 & 0 & 1 & 1 & 0 & 3\\
On qualitative differences in learning: I—Outcome and process,Marton, Ference and S{\"a}lj{\"o}, Roger & 6 & 8 & 0 & 0 & 5 & 0 & 19\\
A subsumption theory of meaningful verbal learning and retention,Ausubel, David P & 0 & 0 & 1 & 0 & 5 & 0 & 6\\
Transformative dimensions of adult learning.,Mezirow, Jack & 0 & 0 & 0 & 0 & 0 & 0 & 0\\
Working memory,Baddeley, Alan & 0 & 0 & 3 & 0 & 0 & 0 & 3\\
Interaction between learning and development,Vygotsky, Lev & 0 & 1 & 0 & 0 & 0 & 0 & 1\\
Some difficulties of learning to program,Du Boulay, Benedict &  &  &  &  &  &  & \\
Overcoming barriers to student understanding: Threshold concepts and troublesome knowledge,Meyer, Jan and Land, Ray &  &  &  &  &  &  & \\
Threshold concepts and threshold skills in computing,Sanders, Kate and Boustedt, Jonas and Eckerdal, Anna and McCartney, Robert and Mostr{\"o}m, Jan Erik and Thomas, Lynda and Zander, Carol &  &  &  &  &  &  & \\
Learning edge momentum: A new account of outcomes in CS1,Robins, Anthony &  &  &  &  &  &  & \\
The zones of proximal flow: guiding students through a space of computational thinking skills and challenges,Basawapatna, Ashok R and Repenning, Alexander and Koh, Kyu Han and Nickerson, Hilarie &  &  &  &  &  &  & \\
Defensive climate in the computer science classroom,Barker, Lecia Jane and Garvin-Doxas, Kathy and Jackson, Michele &  &  &  &  &  &  & \\
Methods and tools for exploring novice compilation behaviour,Jadud, Matthew C &  &  &  &  &  &  & \\
Exploring the role of visualization and engagement in computer science education,Naps, Thomas L and R{\"o}{\ss}ling, Guido and Almstrum, Vicki and Dann, Wanda and Fleischer, Rudolf and Hundhausen, Chris and Korhonen, Ari and Malmi, Lauri and McNally, Myles and Rodger, Susan and others &  &  &  &  &  &  & \\
Spatial learning and reasoning skill,Thorndyke, Perry W and Goldin, Sarah E &  &  &  &  &  &  & \\
A schema theory of discrete motor skill learning.,Schmidt, Richard A &  &  &  &  &  &  & \\
Modeling the learning progressions of computational thinking of primary grade students,Seiter, Linda and Foreman, Brendan &  &  &  &  &  &  & \\
Reducing abstraction level when learning abstract algebra concepts,Hazzan, Orit &  &  &  &  &  &  & \\
Identity and the Life Cycle: Selected Papers,Erikson, Erik H &  &  &  &  &  &  & \\
Studying context: A comparison of activity theory, situated action models, and distributed cognition,Nardi, Bonnie A &  &  &  &  &  &  & \\
Diffusion of innovations.,ROGERS, Everett M and others &  &  &  &  &  &  & \\
Reflexivity: towards a theory of lifelong learning,Edwards, Richard and Ranson, Stewart and Strain, Michael &  &  &  &  &  &  & \\
Block Model: an educational model of program comprehension as a tool for a scholarly approach to teaching,Schulte, Carsten &  &  &  &  &  &  & \\
Self-efficacy: toward a unifying theory of behavioral change.,Bandura, Albert &  &  &  &  &  &  & \\
Grit: perseverance and passion for long-term goals.,Duckworth, Angela L and Peterson, Christopher and Matthews, Michael D and Kelly, Dennis R &  &  &  &  &  &  & \\
A Theory of Debugging Plans and Interpretations.,Simmons, Reid G &  &  &  &  &  &  & \\
Intrinsic motivation,Deci, Edward L and Ryan, Richard M &  &  &  &  &  &  & \\
Misconceptions reconceived: A constructivist analysis of knowledge in transition,Smith III, John P and Disessa, Andrea A and Roschelle, Jeremy &  &  &  &  &  &  & \\
Misconceptions and attitudes that interfere with learning to program,Clancy, Michael &  &  &  &  &  &  & \\
Empirical studies of programming knowledge,Soloway, Elliot and Ehrlich, Kate &  &  &  &  &  &  & \\
A Theory of Debugging Plans and Interpretations.,Simmons, Reid G &  &  &  &  &  &  & \\
Development and evaluation of a model of programming errors,Ko, Andrew Jensen and Myers, Brad A &  &  &  &  &  &  & \\
\end{tabular}
\caption{References to key papers in selected CS Education venues, as identified through Google Scholar.}
\end{table*}