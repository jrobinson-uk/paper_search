\begin{table*}[t]
\begin{tabular}{p{3cm}p{7cm}llllllll}
Theory&Paper&TOCE&CE&ICER&SIGCSE&ITiCSE&Koli&ACE&Total\\\hline
Achievement Goal Theory & \textit{Achievement motivation: Conceptions of ability, subjective experience, task choice, and performance}~\cite{nicholls1984achievement} & 0 & 0 & 0 & 0 & 0 & 0 & 0 & 0\\
Affordance Theory & \textit{The theory of affordances}~\cite{gibson1977theory} & 1 & 0 & 0 & 0 & 1 & 0 & 0 & 2\\
Associative Learning & \textit{The organization of behavior. A neuropsychological theory}~\cite{hebb1949organization} & 0 & 0 & 0 & 0 & 0 & 0 & 0 & 0\\
Analytical Behaviorism & \textit{Psychology as the behaviorist views it}~\cite{watson1913psychology} & 1 & 0 & 0 & 0 & 0 & 0 & 0 & 1\\
Learning Styles & \textit{Learning and teaching styles in engineering education}~\cite{felder1988learning} & 2 & 5 & 0 & 2 & 7 & 4 & 0 & 20\\
Information Processing Theory & \textit{The magical number seven, plus or minus two: Some limits on our capacity for processing information}~\cite{miller1956magical} & 4 & 5 & 2 & 1 & 1 & 3 & 1 & 17\\
Cognitive Apprenticeship & \textit{Cognitive apprenticeship: Teaching the craft of reading, writing and mathematics}~\cite{collins1988cognitive} & 4 & 7 & 2 & 6 & 4 & 2 & 0 & 25\\
Cognitive Load Theory & \textit{Cognitive load during problem solving: Effects on learning}~\cite{sweller1988cognitive} & 2 & 2 & 11 & 2 & 3 & 1 & 1 & 22\\
Cognitive Theory of Multimedia Learning & \textit{Multimedia learning}~\cite{mayer2002multimedia} & 3 & 2 & 7 & 3 & 2 & 3 & 0 & 20\\
Communities of Practice & \textit{Communities of practice: Learning, meaning, and identity}~\cite{wenger1999communities} & 6 & 9 & 8 & 11 & 6 & 2 & 4 & 46\\
Conditions of Learning & \textit{Conditions of learning}~\cite{gagne1965conditions} & 2 & 0 & 0 & 0 & 0 & 0 & 0 & 2\\
Connectionism & \textit{The fundamentals of learning}~\cite{thorndike1932fundamentals} & 0 & 0 & 0 & 0 & 0 & 0 & 0 & 0\\
Connectivism & \textit{Connectivism: A Learning Theory for the Digital Age}~\cite{siemens2004connectivism} & 1 & 0 & 0 & 0 & 0 & 0 & 0 & 1\\
Constructionism  & \textit{Mindstorms: Children, computers, and powerful ideas}~\cite{papert1980mindstorms} & 16 & 24 & 10 & 39 & 14 & 8 & 1 & 112\\
Constructivism & \textit{Experience and education}~\cite{dewey1986experience} & 3 & 0 & 0 & 2 & 4 & 2 & 0 & 11\\
Conversation Theory & \textit{Conversation, cognition and learning: A cybernetic theory and methodology}~\cite{pask1975conversation} & 0 & 0 & 0 & 0 & 0 & 0 & 0 & 0\\
Discovery Learning & \textit{The act of discovery}~\cite{bruner1961act} & 2 & 1 & 0 & 2 & 1 & 0 & 0 & 6\\
Dual Coding Theory & \textit{Mental representations: A dual coding approach}~\cite{paivio1990mental} & 1 & 0 & 0 & 2 & 2 & 0 & 1 & 6\\
Elaboration Theory & \textit{Elaboration theory}~\cite{reigeluth1983elaboration} & 0 & 0 & 0 & 0 & 0 & 0 & 0 & 0\\
Emotional Intelligence & \textit{Emotional intelligence}~\cite{goleman1995emotional} & 0 & 0 & 1 & 0 & 1 & 1 & 0 & 3\\
Expectancy Theory & \textit{Work and motivation}~\cite{vroom1964work} & 1 & 0 & 0 & 0 & 0 & 0 & 0 & 1\\
Experiential Learning & \textit{Experiential learning: Experience as the source of learning and development}~\cite{kolb1984experiential} & 10 & 5 & 3 & 7 & 3 & 7 & 2 & 37\\
Expertise Theory & \textit{The role of deliberate practice in the acquisition of expert performance}~\cite{ericsson1993role} & 0 & 1 & 3 & 5 & 2 & 1 & 1 & 13\\
Flow (Csikszentmihalyi) & \textit{Flow and the psychology of discovery and invention}~\cite{csikszentmihalyi1997flow} & 2 & 0 & 0 & 2 & 0 & 0 & 0 & 4\\
Functional Context Theory & \textit{Functional Context Education. Workshop Resource Notebook}~\cite{sticht1987functional} & 0 & 0 & 0 & 0 & 0 & 0 & 0 & 0\\
Gestalt Theory & \textit{Gestalt theory}~\cite{wertheimer1938gestalt} & 0 & 0 & 0 & 0 & 0 & 0 & 0 & 0\\
Information Processing Theory & \textit{The magical number seven, plus or minus two: Some limits on our capacity for processing information}~\cite{miller1956magical} & 4 & 5 & 2 & 1 & 1 & 3 & 1 & 17\\
Instrumental Conditioning & \textit{The behavior of organisms}~\cite{skinner1938behavior} & 0 & 0 & 0 & 0 & 0 & 0 & 0 & 0\\
Interest (Kintsch) & \textit{Learning from text, levels of comprehension, or: Why anyone would read a story anyway}~\cite{kintsch1980learning} & 0 & 0 & 0 & 0 & 0 & 0 & 0 & 0\\
Interference Theory & \textit{The present status of interference theory}~\cite{postman1961present} & 0 & 0 & 0 & 0 & 0 & 0 & 0 & 0\\
Invitational Education & \textit{What Is Invitational Education and How Does It Work?}~\cite{purkey1991invitational} & 0 & 0 & 0 & 0 & 0 & 0 & 0 & 0\\
Latent Learning & \textit{There is more than one kind of learning}~\cite{tolman1949there} & 2 & 0 & 0 & 0 & 0 & 0 & 0 & 2\\
Learner-Centered Design & \textit{Learner-centered design: The challenge for HCI in the 21st century}~\cite{soloway1994learner} & 0 & 0 & 2 & 1 & 0 & 0 & 0 & 3\\
Learning Styles & \textit{Learning and teaching styles in engineering education}~\cite{felder1988learning} & 2 & 5 & 0 & 2 & 7 & 4 & 0 & 20\\
Maslow’s Hierarchy of Needs & \textit{A theory of human motivation}~\cite{maslow1943theory} & 2 & 0 & 0 & 0 & 1 & 0 & 0 & 3\\
Mastery Learning & \textit{Mastery learning}~\cite{bloom1971mastery} & 1 & 0 & 4 & 2 & 3 & 1 & 1 & 12\\
Mental Models & \textit{Mental models in cognitive science}~\cite{johnson1980mental} & 0 & 0 & 0 & 0 & 0 & 0 & 0 & 0\\
Metacognition & \textit{Metacognition and cognitive monitoring: A new area of cognitive--developmental inquiry}~\cite{flavell1979metacognition} & 9 & 6 & 5 & 13 & 19 & 7 & 3 & 62\\
Mindsets & \textit{A social-cognitive approach to motivation and personality}~\cite{dweck1988social} & 0 & 1 & 3 & 0 & 1 & 1 & 0 & 6\\
Split Attention Effect & \textit{A split-attention effect in multimedia learning: Evidence for dual processing systems in working memory}~\cite{mayer1998split} & 0 & 0 & 1 & 1 & 0 & 0 & 0 & 2\\
Operant Conditioning & \textit{The behavior of organisms: an experimental analysis}~\cite{skinner1938behavior} & 0 & 0 & 0 & 0 & 0 & 0 & 0 & 0\\
Problem-Based Learning  & \textit{Problem-based learning: An approach to medical education}~\cite{barrows1980problem} & 0 & 0 & 0 & 0 & 0 & 0 & 0 & 0\\
Psychological Behaviorism & \textit{Behavior and personality: Psychological behaviorism}~\cite{staats1996behavior} & 0 & 0 & 0 & 0 & 0 & 0 & 0 & 0\\
Rescorla-Wagner Classical Conditioning & \textit{A theory of Pavlovian conditioning: Variations in the effectiveness of reinforcement and nonreinforcement}~\cite{rescorla1972theory} & 0 & 0 & 0 & 0 & 0 & 0 & 0 & 0\\
Scaffolding & \textit{The role of tutoring in problem solving}~\cite{wood1976role} & 1 & 3 & 1 & 1 & 6 & 0 & 0 & 12\\
Self-Determination Theory & \textit{Self-determination theory and the facilitation of intrinsic motivation, social development, and well-being}~\cite{ryan2000self} & 6 & 5 & 1 & 3 & 2 & 1 & 0 & 18\\
Sensory Theory & \textit{Approaches to training and development}~\cite{laird1985approaches} & 0 & 0 & 0 & 0 & 0 & 0 & 0 & 0\\
Situated Cognition & \textit{Situated cognition and the culture of learning}~\cite{brown1989situated} & 6 & 5 & 5 & 6 & 4 & 2 & 1 & 29\\
Situated Learning & \textit{Situated learning: Legitimate peripheral participation}~\cite{lave1991situated} & 14 & 12 & 14 & 20 & 11 & 10 & 1 & 82\\
Social Development Theory & \textit{Mind in society: The development of higher psychological processes}~\cite{vygotsky1978mind} & 16 & 12 & 10 & 23 & 11 & 3 & 1 & 76\\
Social Learning Theory & \textit{Social learning theory}~\cite{bandura1977social} & 0 & 0 & 2 & 1 & 0 & 0 & 0 & 3\\
Split Attention Effect & \textit{A split-attention effect in multimedia learning: Evidence for dual processing systems in working memory}~\cite{mayer1998split} & 0 & 0 & 1 & 1 & 0 & 0 & 0 & 2\\
??? & \textit{The origins of intelligence in children}~\cite{???} & 0 & 0 & 1 & 1 & 2 & 1 & 0 & 5\\
Approaches to Learning (deep, surface) & \textit{On qualitative differences in learning: I—Outcome and process}~\cite{marton1976qualitative} & 7 & 9 & 0 & 4 & 8 & 8 & 0 & 36\\
Subsumption Theory / Reception Learning & \textit{A subsumption theory of meaningful verbal learning and retention}~\cite{ausubel1962subsumption} & 1 & 3 & 1 & 1 & 6 & 0 & 0 & 12\\
Transformative Learning & \textit{Transformative dimensions of adult learning}~\cite{mezirow1991transformative} & 0 & 0 & 0 & 0 & 0 & 0 & 0 & 0\\
Working Memory & \textit{Working memory}~\cite{baddeley1992working} & 0 & 0 & 3 & 0 & 0 & 0 & 0 & 3\\
Zone of Proximal Development & \textit{Interaction between learning and development}~\cite{vygotsky1978interaction} & 0 & 1 & 0 & 2 & 0 & 1 & 0 & 4\\
Notional Machine & \textit{Some difficulties of learning to program}~\cite{du1986some} & 9 & 14 & 17 & 10 & 10 & 9 & 1 & 70\\
Threshold Concepts & \textit{Overcoming barriers to student understanding: Threshold concepts and troublesome knowledge}~\cite{meyer2006overcoming} & 2 & 1 & 0 & 0 & 1 & 2 & 0 & 6\\
Threshold Skills & \textit{Threshold concepts and threshold skills in computing}~\cite{sanders2012threshold} & 2 & 2 & 1 & 1 & 2 & 2 & 0 & 10\\
Learning Edge Momentum & \textit{Learning edge momentum: A new account of outcomes in CS1}~\cite{robins2010learning} & 4 & 2 & 6 & 16 & 6 & 4 & 3 & 41\\
Proximal Flow & \textit{The zones of proximal flow: guiding students through a space of computational thinking skills and challenges}~\cite{basawapatna2013zones} & 1 & 2 & 0 & 4 & 3 & 0 & 0 & 10\\
Defensive Climate & \textit{Defensive climate in the computer science classroom}~\cite{barker2002defensive} & 3 & 4 & 1 & 3 & 1 & 1 & 0 & 13\\
EQ & \textit{Methods and tools for exploring novice compilation behaviour}~\cite{jadud2006methods} & 8 & 3 & 13 & 12 & 12 & 5 & 2 & 55\\
Engagement Taxonomy & \textit{Exploring the role of visualization and engagement in computer science education}~\cite{naps2002exploring} & 18 & 6 & 2 & 17 & 13 & 19 & 1 & 76\\
Spatial Reasoning Skills & \textit{Spatial learning and reasoning skill}~\cite{thorndyke1983spatial} & 0 & 0 & 0 & 0 & 0 & 0 & 0 & 0\\
Schema Theory & \textit{A schema theory of discrete motor skill learning}~\cite{schmidt1975schema} & 0 & 0 & 0 & 0 & 0 & 0 & 0 & 0\\
PECT & \textit{Modeling the learning progressions of computational thinking of primary grade students}~\cite{seiter2013modeling} & 0 & 1 & 11 & 6 & 1 & 0 & 0 & 19\\
Reducing Abstraction & \textit{Reducing abstraction level when learning abstract algebra concepts}~\cite{hazzan1999reducing} & 1 & 4 & 0 & 0 & 1 & 1 & 0 & 7\\
Identity Formation & \textit{Identity and the Life Cycle: Selected Papers}~\cite{erikson1968identity} & 0 & 0 & 0 & 0 & 0 & 0 & 0 & 0\\
Contextualized Computing & \textit{Studying context: A comparison of activity theory, situated action models, and distributed cognition}~\cite{nardi1996studying} & 1 & 0 & 0 & 0 & 0 & 0 & 0 & 1\\
Innovation Diffusion & \textit{Diffusion of innovations}~\cite{rogers1962diffusion} & 1 & 1 & 0 & 3 & 1 & 1 & 1 & 8\\
Learning Trajectories & \textit{Reflexivity: towards a theory of lifelong learning}~\cite{edwards2002reflexivity} & 0 & 0 & 0 & 0 & 0 & 0 & 0 & 0\\

Block Model & \textit{Block Model: an educational model of program comprehension as a tool for a scholarly approach to teaching}~\cite{schulte2008block} & 0 & 3 & 3 & 3 & 3 & 2 & 0 & 14\\
Self-efficacy & \textit{Self-efficacy: toward a unifying theory of behavioral change}~\cite{bandura1977self} & 5 & 5 & 6 & 9 & 4 & 1 & 0 & 30\\
Grit & \textit{Grit: perseverance and passion for long-term goals}~\cite{duckworth2007grit} & 1 & 0 & 1 & 0 & 2 & 1 & 0 & 5\\
Debugging Plans & \textit{A Theory of Debugging Plans and Interpretations}~\cite{simmons1988theory} & 0 & 0 & 0 & 0 & 0 & 0 & 0 & 0\\
Motivation Theories & \textit{Intrinsic motivation}~\cite{deci2010intrinsic} & 3 & 3 & 1 & 1 & 1 & 0 & 0 & 9\\
Geek Gene & \textit{Predicting Student Success in an Introductory Programming Course}~\cite{Hostetler:1983} & 0 & 3 & 1 & 1 & 1 & 0 & 0 & 6\\
Errors / Misconceptions & \textit{Misconceptions reconceived: A constructivist analysis of knowledge in transition}~\cite{smith1994misconceptions} & 1 & 6 & 2 & 2 & 1 & 1 & 0 & 13\\
Attitudes & \textit{Misconceptions and attitudes that interfere with learning to program}~\cite{clancy2004misconceptions} & 7 & 5 & 3 & 5 & 4 & 3 & 0 & 27\\
Programming Plans & \textit{Empirical studies of programming knowledge}~\cite{soloway1984empirical} & 0 & 10 & 5 & 0 & 2 & 4 & 0 & 21\\
Debugging Plans & \textit{A Theory of Debugging Plans and Interpretations}~\cite{simmons1988theory} & 0 & 0 & 0 & 0 & 0 & 0 & 0 & 0\\
Model of Programming Errors & \textit{Development and evaluation of a model of programming errors}~\cite{ko2003development} & 1 & 0 & 0 & 0 & 0 & 0 & 0 & 1\\
\end{tabular}
\caption{References to key papers in selected CS Education venues, as identified through Google Scholar.}
\end{table*}