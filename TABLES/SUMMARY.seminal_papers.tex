\begin{table*}[t]
\begin{tabular}{p{3cm}p{7cm}lllllll}
Theory&Paper&TOCE&Koli&ICER&SIGCSE&ITiCSE&ACE&Total\\\hline
Achievement Goal Theory & \textit{Achievement motivation: Conceptions of ability, subjective experience, task choice, and performance} & 0 & 0 & 0 & 0 & 0 & 0 & 0\\
Affordance Theory & \textit{The theory of affordances} & 1 & 0 & 0 & 0 & 1 & 0 & 2\\
Associative Learning & \textit{The organization of behavior. A neuropsychological theory} & 0 & 0 & 0 & 0 & 0 & 0 & 0\\
Analytical Behaviorism & \textit{Psychology as the behaviorist views it} & 1 & 0 & 0 & 0 & 0 & 0 & 1\\
Learning Styles & \textit{Learning and teaching styles in engineering education} & 1 & 4 & 0 & 0 & 7 & 0 & 12\\
Information Processing Theory & \textit{The magical number seven, plus or minus two: Some limits on our capacity for processing information} & 3 & 3 & 2 & 0 & 0 & 1 & 9\\
Cognitive Apprenticeship & \textit{Cognitive apprenticeship: Teaching the craft of reading, writing and mathematics} & 4 & 2 & 2 & 0 & 3 & 0 & 11\\
Cognitive Load Theory & \textit{Cognitive load during problem solving: Effects on learning} & 2 & 1 & 11 & 0 & 2 & 1 & 17\\
Cognitive Theory of Multimedia Learning & \textit{Multimedia learning} & 3 & 3 & 6 & 0 & 1 & 0 & 13\\
Communities of Practice & \textit{Communities of practice: Learning, meaning, and identity} & 5 & 2 & 7 & 2 & 4 & 4 & 24\\
Conditions of Learning & \textit{Conditions of learning} & 1 & 0 & 0 & 0 & 0 & 0 & 1\\
Connectionism & \textit{The fundamentals of learning} & 0 & 0 & 0 & 0 & 0 & 0 & 0\\
Connectivism & \textit{Connectivism: A Learning Theory for the Digital Age} & 0 & 0 & 0 & 0 & 0 & 0 & 0\\
Constructionism  & \textit{Mindstorms: Children, computers, and powerful ideas} & 15 & 8 & 10 & 5 & 12 & 1 & 51\\
Constructivism & \textit{Experience and education} & 3 & 2 & 0 & 1 & 4 & 0 & 10\\
Conversation Theory & \textit{Conversation, cognition and learning: A cybernetic theory and methodology} & 0 & 0 & 0 & 0 & 0 & 0 & 0\\
Discovery Learning & \textit{The act of discovery} & 2 & 0 & 0 & 0 & 0 & 0 & 2\\
Dual Coding Theory & \textit{Mental representations: A dual coding approach} & 1 & 0 & 0 & 0 & 2 & 1 & 4\\
Elaboration Theory & \textit{Elaboration theory} & 0 & 0 & 0 & 0 & 0 & 0 & 0\\
Emotional Intelligence & \textit{Emotional intelligence} & 0 & 1 & 1 & 0 & 1 & 0 & 3\\
Expectancy Theory & \textit{Work and motivation} & 1 & 0 & 0 & 0 & 0 & 0 & 1\\
Experiential Learning & \textit{Experiential learning: Experience as the source of learning and development} & 7 & 7 & 2 & 1 & 3 & 2 & 22\\
Expertise Theory & \textit{The role of deliberate practice in the acquisition of expert performance} & 0 & 1 & 3 & 1 & 1 & 1 & 7\\
Flow (Csikszentmihalyi) & \textit{Flow and the psychology of discovery and invention} & 2 & 0 & 0 & 1 & 0 & 0 & 3\\
Functional Context Theory & \textit{Functional Context Education. Workshop Resource Notebook} & 0 & 0 & 0 & 1 & 0 & 0 & 1\\
Gestalt Theory & \textit{Gestalt theory} & 0 & 0 & 0 & 0 & 0 & 0 & 0\\
Information Processing Theory & \textit{The magical number seven, plus or minus two: Some limits on our capacity for processing information} & 3 & 3 & 2 & 0 & 0 & 1 & 9\\
Instrumental Conditioning & \textit{The behavior of organisms} & 0 & 0 & 0 & 0 & 0 & 0 & 0\\
Interest (Kintsch) & \textit{Learning from text, levels of comprehension, or: Why anyone would read a story anyway} & 0 & 0 & 0 & 0 & 0 & 0 & 0\\
Interference Theory & \textit{The present status of interference theory} & 0 & 0 & 0 & 0 & 0 & 0 & 0\\
Invitational Education & \textit{What Is Invitational Education and How Does It Work?} & 0 & 0 & 0 & 0 & 0 & 0 & 0\\
Latent Learning & \textit{There is more than one kind of learning} & 1 & 0 & 0 & 0 & 0 & 0 & 1\\
Learner-Centered Design & \textit{Learner-centered design: The challenge for HCI in the 21st century} & 0 & 0 & 2 & 0 & 0 & 0 & 2\\
Learning Styles & \textit{Learning and teaching styles in engineering education} & 1 & 4 & 0 & 0 & 7 & 0 & 12\\
Maslow’s Hierarchy of Needs & \textit{A theory of human motivation} & 2 & 0 & 0 & 0 & 1 & 0 & 3\\
Mastery Learning & \textit{Mastery learning} & 1 & 1 & 4 & 0 & 2 & 1 & 9\\
Mental Models & \textit{Mental models in cognitive science} & 0 & 0 & 0 & 0 & 0 & 0 & 0\\
Metacognition & \textit{Metacognition and cognitive monitoring: A new area of cognitive--developmental inquiry} & 9 & 7 & 4 & 0 & 16 & 3 & 39\\
Mindsets & \textit{A social-cognitive approach to motivation and personality} & 0 & 1 & 3 & 0 & 1 & 0 & 5\\
Split Attention Effect & \textit{A split-attention effect in multimedia learning: Evidence for dual processing systems in working memory} & 0 & 0 & 1 & 0 & 0 & 0 & 1\\
Operant Conditioning & \textit{The behavior of organisms: an experimental analysis} & 0 & 0 & 0 & 0 & 0 & 0 & 0\\
Problem-Based Learning  & \textit{Problem-based learning: An approach to medical education} & 0 & 0 & 0 & 0 & 0 & 0 & 0\\
Psychological Behaviorism & \textit{Behavior and personality: Psychological behaviorism} & 0 & 0 & 0 & 0 & 0 & 0 & 0\\
Rescorla-Wagner Classical Conditioning & \textit{A theory of Pavlovian conditioning: Variations in the effectiveness of reinforcement and nonreinforcement} & 0 & 0 & 0 & 0 & 0 & 0 & 0\\
Scaffolding & \textit{The role of tutoring in problem solving} & 0 & 0 & 1 & 0 & 5 & 0 & 6\\
Self-Determination Theory & \textit{Self-determination theory and the facilitation of intrinsic motivation, social development, and well-being} & 6 & 1 & 1 & 0 & 2 & 0 & 10\\
Sensory Theory & \textit{Approaches to training and development} & 0 & 0 & 0 & 0 & 0 & 0 & 0\\
Situated Cognition & \textit{Situated cognition and the culture of learning} & 5 & 2 & 4 & 1 & 4 & 1 & 17\\
Situated Learning & \textit{Situated learning: Legitimate peripheral participation} & 13 & 10 & 12 & 2 & 9 & 1 & 47\\
Social Development Theory & \textit{Mind in society: The development of higher psychological processes} & 15 & 3 & 9 & 1 & 7 & 1 & 36\\
Social Learning Theory & \textit{Social learning theory} & 0 & 0 & 1 & 0 & 0 & 0 & 1\\
Split Attention Effect & \textit{A split-attention effect in multimedia learning: Evidence for dual processing systems in working memory} & 0 & 0 & 1 & 0 & 0 & 0 & 1\\
??? & \textit{The origins of intelligence in children} & 0 & 1 & 0 & 1 & 1 & 0 & 3\\
Approaches to Learning (deep, surface) & \textit{On qualitative differences in learning: I—Outcome and process} & 6 & 8 & 0 & 0 & 5 & 0 & 19\\
Subsumption Theory / Reception Learning & \textit{A subsumption theory of meaningful verbal learning and retention} & 0 & 0 & 1 & 0 & 5 & 0 & 6\\
Transformative Learning & \textit{Transformative dimensions of adult learning} & 0 & 0 & 0 & 0 & 0 & 0 & 0\\
Working Memory & \textit{Working memory} & 0 & 0 & 3 & 0 & 0 & 0 & 3\\
Zone of Proximal Development & \textit{Interaction between learning and development} & 0 & 1 & 0 & 0 & 0 & 0 & 1\\
Notional Machine & \textit{Some difficulties of learning to program} &  &  &  &  &  &  & \\
Threshold Concepts & \textit{Overcoming barriers to student understanding: Threshold concepts and troublesome knowledge} &  &  &  &  &  &  & \\
Threshold Skills & \textit{Threshold concepts and threshold skills in computing} &  &  &  &  &  &  & \\
Learning Edge Momentum & \textit{Learning edge momentum: A new account of outcomes in CS1} &  &  &  &  &  &  & \\
Proximal Flow & \textit{The zones of proximal flow: guiding students through a space of computational thinking skills and challenges} &  &  &  &  &  &  & \\
Defensive Climate & \textit{Defensive climate in the computer science classroom} &  &  &  &  &  &  & \\
EQ & \textit{Methods and tools for exploring novice compilation behaviour} &  &  &  &  &  &  & \\
Engagement Taxonomy & \textit{Exploring the role of visualization and engagement in computer science education} &  &  &  &  &  &  & \\
spatial reasoning skills (someone please check this one, I'm not sure I'm pulling in the correct reference here as it is not CS specific) & \textit{Spatial learning and reasoning skill} &  &  &  &  &  &  & \\
Schema Theory & \textit{A schema theory of discrete motor skill learning} &  &  &  &  &  &  & \\
PECT & \textit{Modeling the learning progressions of computational thinking of primary grade students} &  &  &  &  &  &  & \\
Reducing Abstraction & \textit{Reducing abstraction level when learning abstract algebra concepts} &  &  &  &  &  &  & \\
Identity Formation & \textit{Identity and the Life Cycle: Selected Papers} &  &  &  &  &  &  & \\
Contextualized Computing & \textit{Studying context: A comparison of activity theory, situated action models, and distributed cognition} &  &  &  &  &  &  & \\
Innovation Diffusion & \textit{Diffusion of innovations} &  &  &  &  &  &  & \\
Learning Trajectories & \textit{Reflexivity: towards a theory of lifelong learning} &  &  &  &  &  &  & \\

Block Model & \textit{Block Model: an educational model of program comprehension as a tool for a scholarly approach to teaching} &  &  &  &  &  &  & \\
Self-efficacy & \textit{Self-efficacy: toward a unifying theory of behavioral change} &  &  &  &  &  &  & \\
Grit & \textit{Grit: perseverance and passion for long-term goals} &  &  &  &  &  &  & \\
Debugging Plans & \textit{A Theory of Debugging Plans and Interpretations} &  &  &  &  &  &  & \\
Motivation Theories & \textit{Intrinsic motivation} &  &  &  &  &  &  & \\
Errors / Misconceptions & \textit{Misconceptions reconceived: A constructivist analysis of knowledge in transition} &  &  &  &  &  &  & \\
Attitudes & \textit{Misconceptions and attitudes that interfere with learning to program} &  &  &  &  &  &  & \\
Programming Plans & \textit{Empirical studies of programming knowledge} &  &  &  &  &  &  & \\
Debugging Plans & \textit{A Theory of Debugging Plans and Interpretations} &  &  &  &  &  &  & \\
Model of Programming Errors & \textit{Development and evaluation of a model of programming errors} &  &  &  &  &  &  & \\
\end{tabular}
\caption{References to key papers in selected CS Education venues, as identified through Google Scholar.}
\end{table*}