\begin{table*}[t]
\begin{tabular}{p{10cm}lllllll}
Paper&TOCE&Koli&ICER&SIGCSE&ITiCSE&ACE&Total\\\hline
Achievement motivation: Conceptions of ability, subjective experience, task choice, and performance. & 0 & 0 & 0 & 0 & 0 & 0 & 0\\
The theory of affordances & 1 & 0 & 0 & 0 & 1 & 0 & 2\\
The organization of behavior. A neuropsychological theory & 0 & 0 & 0 & 0 & 0 & 0 & 0\\
Psychology as the behaviorist views it. & 1 & 0 & 0 & 0 & 0 & 0 & 1\\
Learning and teaching styles in engineering education & 1 & 4 & 0 & 0 & 7 & 0 & 12\\
The magical number seven, plus or minus two: Some limits on our capacity for processing information. & 3 & 3 & 2 & 0 & 0 & 1 & 9\\
Cognitive apprenticeship: Teaching the craft of reading, writing and mathematics & 4 & 2 & 2 & 0 & 3 & 0 & 11\\
Cognitive load during problem solving: Effects on learning & 2 & 1 & 11 & 0 & 2 & 1 & 17\\
Multimedia learning & 3 & 3 & 6 & 0 & 1 & 0 & 13\\
Communities of practice: Learning, meaning, and identity & 5 & 2 & 7 & 2 & 4 & 4 & 24\\
Conditions of learning & 1 & 0 & 0 & 0 & 0 & 0 & 1\\
The fundamentals of learning. & 0 & 0 & 0 & 0 & 0 & 0 & 0\\
Connectivism: A Learning Theory for the Digital Age & 0 & 0 & 0 & 0 & 0 & 0 & 0\\
Mindstorms: Children, computers, and powerful ideas & 15 & 8 & 10 & 5 & 12 & 1 & 51\\
Experience and education & 3 & 2 & 0 & 1 & 4 & 0 & 10\\
Conversation, cognition and learning: A cybernetic theory and methodology & 0 & 0 & 0 & 0 & 0 & 0 & 0\\
The act of discovery & 2 & 0 & 0 & 0 & 0 & 0 & 2\\
Mental representations: A dual coding approach & 1 & 0 & 0 & 0 & 2 & 1 & 4\\
Elaboration theory & 0 & 0 & 0 & 0 & 0 & 0 & 0\\
Emotional intelligence & 0 & 1 & 1 & 0 & 1 & 0 & 3\\
Work and motivation & 1 & 0 & 0 & 0 & 0 & 0 & 1\\
Experiential learning: Experience as the source of learning and development & 7 & 7 & 2 & 1 & 3 & 2 & 22\\
The role of deliberate practice in the acquisition of expert performance. & 0 & 1 & 3 & 1 & 1 & 1 & 7\\
Flow and the psychology of discovery and invention & 2 & 0 & 0 & 1 & 0 & 0 & 3\\
Functional Context Education. Workshop Resource Notebook. & 0 & 0 & 0 & 1 & 0 & 0 & 1\\
Gestalt theory. & 0 & 0 & 0 & 0 & 0 & 0 & 0\\
The magical number seven, plus or minus two: Some limits on our capacity for processing information. & 3 & 3 & 2 & 0 & 0 & 1 & 9\\
The behavior of organisms & 0 & 0 & 0 & 0 & 0 & 0 & 0\\
Learning from text, levels of comprehension, or: Why anyone would read a story anyway & 0 & 0 & 0 & 0 & 0 & 0 & 0\\
The present status of interference theory. & 0 & 0 & 0 & 0 & 0 & 0 & 0\\
What Is Invitational Education and How Does It Work?. & 0 & 0 & 0 & 0 & 0 & 0 & 0\\
There is more than one kind of learning. & 1 & 0 & 0 & 0 & 0 & 0 & 1\\
Learner-centered design: The challenge for HCI in the 21st century & 0 & 0 & 2 & 0 & 0 & 0 & 2\\
Learning and teaching styles in engineering education & 1 & 4 & 0 & 0 & 7 & 0 & 12\\
A theory of human motivation. & 2 & 0 & 0 & 0 & 1 & 0 & 3\\
Mastery learning & 1 & 1 & 4 & 0 & 2 & 1 & 9\\
Mental models in cognitive science & 0 & 0 & 0 & 0 & 0 & 0 & 0\\
Metacognition and cognitive monitoring: A new area of cognitive--developmental inquiry. & 9 & 7 & 4 & 0 & 16 & 3 & 39\\
A social-cognitive approach to motivation and personality. & 0 & 1 & 3 & 0 & 1 & 0 & 5\\
A split-attention effect in multimedia learning: Evidence for dual processing systems in working memory. & 0 & 0 & 1 & 0 & 0 & 0 & 1\\
The behavior of organisms: an experimental analysis. & 0 & 0 & 0 & 0 & 0 & 0 & 0\\
Problem-based learning: An approach to medical education & 0 & 0 & 0 & 0 & 0 & 0 & 0\\
Behavior and personality: Psychological behaviorism & 0 & 0 & 0 & 0 & 0 & 0 & 0\\
A theory of Pavlovian conditioning: Variations in the effectiveness of reinforcement and nonreinforcement & 0 & 0 & 0 & 0 & 0 & 0 & 0\\
The role of tutoring in problem solving & 0 & 0 & 1 & 0 & 5 & 0 & 6\\
Self-determination theory and the facilitation of intrinsic motivation, social development, and well-being. & 6 & 1 & 1 & 0 & 2 & 0 & 10\\
Approaches to training and development & 0 & 0 & 0 & 0 & 0 & 0 & 0\\
Situated cognition and the culture of learning & 5 & 2 & 4 & 1 & 4 & 1 & 17\\
Situated learning: Legitimate peripheral participation & 13 & 10 & 12 & 2 & 9 & 1 & 47\\
Mind in society: The development of higher psychological processes & 15 & 3 & 9 & 1 & 7 & 1 & 36\\
Social learning theory & 0 & 0 & 1 & 0 & 0 & 0 & 1\\
A split-attention effect in multimedia learning: Evidence for dual processing systems in working memory. & 0 & 0 & 1 & 0 & 0 & 0 & 1\\
The origins of intelligence in children & 0 & 1 & 0 & 1 & 1 & 0 & 3\\
On qualitative differences in learning: I—Outcome and process & 6 & 8 & 0 & 0 & 5 & 0 & 19\\
A subsumption theory of meaningful verbal learning and retention & 0 & 0 & 1 & 0 & 5 & 0 & 6\\
Transformative dimensions of adult learning. & 0 & 0 & 0 & 0 & 0 & 0 & 0\\
Working memory & 0 & 0 & 3 & 0 & 0 & 0 & 3\\
Interaction between learning and development & 0 & 1 & 0 & 0 & 0 & 0 & 1\\
Some difficulties of learning to program &  &  &  &  &  &  & \\
Overcoming barriers to student understanding: Threshold concepts and troublesome knowledge &  &  &  &  &  &  & \\
Threshold concepts and threshold skills in computing &  &  &  &  &  &  & \\
Learning edge momentum: A new account of outcomes in CS1 &  &  &  &  &  &  & \\
The zones of proximal flow: guiding students through a space of computational thinking skills and challenges &  &  &  &  &  &  & \\
Defensive climate in the computer science classroom &  &  &  &  &  &  & \\
Methods and tools for exploring novice compilation behaviour &  &  &  &  &  &  & \\
Exploring the role of visualization and engagement in computer science education &  &  &  &  &  &  & \\
Spatial learning and reasoning skill &  &  &  &  &  &  & \\
A schema theory of discrete motor skill learning. &  &  &  &  &  &  & \\
Modeling the learning progressions of computational thinking of primary grade students &  &  &  &  &  &  & \\
Reducing abstraction level when learning abstract algebra concepts &  &  &  &  &  &  & \\
Identity and the Life Cycle: Selected Papers &  &  &  &  &  &  & \\
Studying context: A comparison of activity theory, situated action models, and distributed cognition &  &  &  &  &  &  & \\
Diffusion of innovations. &  &  &  &  &  &  & \\
Reflexivity: towards a theory of lifelong learning &  &  &  &  &  &  & \\
Block Model: an educational model of program comprehension as a tool for a scholarly approach to teaching &  &  &  &  &  &  & \\
Self-efficacy: toward a unifying theory of behavioral change. &  &  &  &  &  &  & \\
Grit: perseverance and passion for long-term goals. &  &  &  &  &  &  & \\
A Theory of Debugging Plans and Interpretations. &  &  &  &  &  &  & \\
Intrinsic motivation &  &  &  &  &  &  & \\
Misconceptions reconceived: A constructivist analysis of knowledge in transition &  &  &  &  &  &  & \\
Misconceptions and attitudes that interfere with learning to program &  &  &  &  &  &  & \\
Empirical studies of programming knowledge &  &  &  &  &  &  & \\
A Theory of Debugging Plans and Interpretations. &  &  &  &  &  &  & \\
Development and evaluation of a model of programming errors &  &  &  &  &  &  & \\
\end{tabular}
\caption{References to key papers in selected CS Education venues, as identified through Google Scholar.}
\end{table*}